
Cities are dynamic. They experience daily, weekly, and yearly rhythms of human activity and travel---while in the longer run, they undergo changes in their urban form, transportation networks, and social geographies. This dissertation studies how these long-term changes can impact residents' well being. It specifically examines trends of growing rates of poverty in the suburbs of Canadian cities during the late 20th and early 21st centuries, and the implications of such to daily travel behaviour.


% cities good, suburbs bad ??
Contemporary human history is one of urbanization. More than 50\% of the world's population lives within urban regions. This tops 80\% in many developed nations, including Canada. Urbanization is associated with overall levels of poverty reduction, economic growth, increases in life expectancy, and improvements in a number of other social indicators; while concurrently, modern urban development exerts a great toll on the natural environments and can exasperate socio-economic inequalities \cite{whitmee_safeguarding_2015}. These negative outcomes can partly be attributed to the qualities of urban development, particularly the suburban auto-oriented environments built over the past century. While offering quality of life improvements for many, suburbanization has consumed thousands of hectares of natural land as well as increased per capita energy use and GHG emissions \cite{ewing_compactness_2015}. It is also well founded that auto-oriented suburban built environments are detrimental to individual health and well-being relative to traditional, more compact, neighbourhoods \shortcite{ewing_relationship_2003}. This has led many researchers to design and advocate for policy and planning strategies that aim to curb urban sprawl and promote more compact development.


% suburbs and transport poverty
Under their veneer of homogeneity, the suburbs can exasperate inequalities and risks of social exclusion. These risks are less studied relative to the environmental and health impacts of suburbanization. Well functioning urban systems provide people the ability for people to participate in daily activities and access destinations necessary for individual health and well-being. However, many suffer from transport poverty, where transport disadvantage (e.g. insufficient public transit options) compounds with other forms of social disadvantage (e.g. low-income or poor health) resulting in barriers to daily travel and thus dissuasion or inability to participate in daily activities, increasing the risks of social exclusion \cite{lucas_transport_2012}. Social exclusion refers to when people are prevented from participating in society and activities necessary for their well-being \cite{levitas_multi-dimensional_2007}. Risks of transport poverty are greatest in suburban environments where cars are costly yet necessary for many \cite{allen_sizing_2019}. In Canada, transport poverty is evident via transport disadvantage being associated with low activity participation rates, particularly for those of lower socio-economic status living in areas of low transit accessibility \shortcite{paez_mobility_2009,roorda_trip_2010,allen_planning_2020}. 


% urban dyamics
Moreover, the demographic and socio-economic patterns of cities are dynamic. Historically, the development of early industrializing cities were mainly monocentric, based on manufacturing, industry, and jobs in the core, and successions of wealth moving out to the suburbs along major transport corridors. Urban segregation and inequalities were thus usually visible via core-periphery gradients \cite{burgess_growth_1925,alonso_location_1964,glaeser_sprawl_2004}. In the later half of the 20th century, there have been substantial population dynamics, which has been framed under post-Fordist restructuring \cite{walks_social_2001}, polycentric development \cite{anas_urban_1998}, and demographic inversions \cite{ehrenhalt_great_2012} such as gentrification \cite{vigdor_does_2002} and suburbanization of poverty \cite{ades_are_2012}. Many contemporary cities, including several in Canada, have evolved into forms where poverty is most prevalent within the inner-suburban ring, located between wealthier older neighbourhoods and outer more recently built suburbs.


% urban change and transport poverty
These recent demographic inversions are concerning from a transport geography perspective as they likely are increasing travel barriers and abetting the scale of transport poverty due to an increased propensity of lower-SES households living in inaccessible neighbourhoods. Despite these trends, there is limited research on the relation between changing urban socio-economic patterns like suburbanization of poverty, differential patterns of urban form and transit (in)accessibility, and risks of transport-related social exclusion. How and where are trends of suburbanization of poverty increasing risks of transport poverty?---Who are the suburban poor? Where do they come from?---Are lower-income residents disproportionately moving away from adequate transit service? These questions have implications for both theory on how the structure and patterns of cities evolve and change over time, as well as for providing evidence to inform policy aimed at poverty reduction and making cities more sustainable. As such, these are the main research questions that guide the empirical chapters of this dissertation.

% research overview

Overall, my research is about human environments at a meso-scale, the scale of socio-spatial interactions, daily travel and activity participation, from the neighbourhood to the region. This consists of a combination of changing built environment, transport networks, the demographic and socio-economic characteristics of residents, and their daily activity patterns. My research is empirically focused on urban regions Canada. From a temporal perspective, my research is primarily retrospective, using historical data to examine trends from the late 20th century to the early 21st century. Methods are mainly quantitative based on survey and administrative data from Statistics Canada, regional travel studies, as well as land use and transport network data from a variety of sources. 

This dissertation consists of six chapters. Chapter \ref{ch:background} provides contextual framing. The goal of this Chapter is not to provide a systematic literature review on a specific topic, but to broadly cover several topics that directly relate to the empirical chapters that follow. This includes a built environment history of suburbanaization in Canada, describing changing social geographies of Canadian cities (i.e. gentrification and subrubanization of poverty), literature on neighbourhood effects and daily travel behaviour, and the relationship between transportation planning and urban dynamics. These sub-sections include both summarizing theoretical and empirical research on these topics as well as their importance in the context of Canadian cities.

Chapter \ref{ch:subtrapov} is the first empirical chapter of the dissertation. It is framed around suburbanization of poverty in Toronto and it's implications for transport poverty. Specifically, it examines how trends of low-income populations concentrating in more automobile oriented areas are related to increased barriers to daily travel and activity participation, particularly for those who are unable to afford a private vehicle. This is examined first from a theoretical perspective, and second via a neighbourhood-level spatio-temporal analysis for the Toronto region from 1991 to 2016. Findings from this chapter show that many suburban areas in Toronto are not only declining in socio-economic status, but are also suffering from increased barriers to daily travel evidenced by longer commute times and decreasing activity participation rates, relative to central neighbourhoods. Because of these adverse effects, this chapter further supports the need for progressive planning and policy aimed at curbing continuing trends of suburbanization of poverty while also improving levels of transport accessibility in the suburbs.

Chapter \ref{ch:pathsubpov} is also about understanding suburbanization of poverty, but instead of a neighbourhood-level analysis, focuses on individual residents and households. The objective of this paper is to describe and quantify the propensity of different individual geographic pathways to becoming poor and living in the suburbs. Specifically, whether suburban poverty primarily a result of 1) residential sorting within cities of low-income residents from central to suburban neighbourhoods (e.g. gentrification and displacement), 2) patterns of immigrant settlement in cities directed towards the suburbs, or 3) becoming and remaining poor while staying in the suburbs?  We do so via a cluster analysis of census and land-use data to define typologies of suburban neighbourhoods and then link this categorization to a panel dataset representing 20\% of tax filers in Canada (from 2006 to 2016). These data allow for analyzing how different pathways are observed within the context of large Canadian cities, i.e. to what extent poverty in suburban neighbourhoods stems from intra-urban residential mobility, external immigration, and becoming poor in-place. Overall, results show that becoming poor while staying in the suburbs encompasses a greater proportion of suburban poverty than immigration and centre-to-suburb residential mobility combined. The chapter is concluded by discussing how this research provides insight into the changing structure of urban neighbourhoods while also providing pertinent information to aid preventative policy aimed at reducing suburban poverty in Canada.

Chapter 5 uses the same data as the previous chapter, but it is directly focused on moves away from public transit in Toronto. Lower-income residents are often reliant on public transit for daily travel. They have also historically concentrated in centrally located neighbourhoods with relatively higher levels of transit accessibility. However, suburbanization of poverty has raised concerns that low-income residents are disproportionately moving away from relied upon transit service. This chapter thus investigates what occurs when low-income residents change dwellings within a region: do they experience a reduction in their levels of transit accessibility, how does this compare to high-income movers, and how has this changed over time? This is examined using the same database of tax filers, subset for Toronto from 1988 to 2018. This data is linked to historical transit accessibility measures (those used in Chapter \ref{ch:subtrapov}), to describe changes in transit accessibility for intra-urban movers via descriptive statistics and regression models to answer whether there are significant differences in individual changes in transit accessibility by low-income status. Findings indicate that low-income residents do, on average, experience reductions in their levels of transit accessibility when moving within the region, but they do not undergo as great of a reduction in transit accessibility as high-income movers. However, the gap in experienced changes in transit accessibility between high- and low-income movers is converging over time.

Chapter \ref{ch:conc} concludes the research by summarizing it's contribution, it's implications for planning and policy, and offering directions for future work. Research is the practice of finding and communicating knowledge and evidence. Indeed, a long noted goal of science is to try to explain nature, based on empirical observation---deciphering how and why things occur, and in geography, the how and why of where. Evidence regarding human environments in particular will almost always have normative interpretations as we collectively are embodied and reproduce our environments. Evidence in this domain does not only explain, but is also used as justification for policy, planning, and design aimed at re-shaping our environments. As such, the concluding chapter includes discussion on the implications of findings both in terms of academic theory and policy recommendations concerning progressive options for promoting environmental sustainability, planetary health, and social inclusion.












