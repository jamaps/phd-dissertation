

\section{A Brief History of Suburbanization}

The changing social and physical patterns of urban spatial structure can be described as a result of competition for desirable space (demand), available land and property (supply), and the development of transportation technologies and networks \cite{alonso_location_1964,anas_urban_1998}. It is the interplay between these, and their dynamics, that has shaped the suburbs and manifested their negative environmental and social impacts.

Cities are often dichotomized between their centre and peripheries. Indeed the suburbs are not a modern phenomena, if we consider the suburbs as simply the transition zone between the urban and rural. The term originates from the Latin \textit{suburbium} pertaining to what is below the city walls. In some ancient cities, these suburban lands housed people that were too poor to live in the city walls, or provided space for the rich and their leisure activities. Being poor and outside the city walls (sub-urban) meant foregoing the protection and utilities because they could not afford it. Historical cities were also often dense places rife with inequality. The elite had space, while the poor were crammed into poor housing. High density cities were necessary given lack of fast transport and often needing to defend behind walls \cite{bruegmann_sprawl:_2008}.

% early suburbanization
The industrial revolution had several major consequences on the pathway towards contemporary forms of (sub)urbanization. Factories and industry tended to locate in the city centre, near major transport hubs. This lead to urban migration, people locating near the available jobs that industry offered. These facilities also worsened the livability of the central city because of their noise and pollution. The industrial revolution also lead to the development of new transportation technologies, such as trains and streetcars. This allowed for the wealthy to live at further distances from the city centre, but still within commuting distance. These trends continued during the early 20th century (e.g. in the interwar years, London's population increased by 10\%, but its land area increased by 20\%). This urban growth was seen as combinations of \textit{aggregations} of urban populations and \textit{expansion} of urbanized land (i.e. expansion and intensification, concentration and de-centralization), that manifested itself in decreasing density and increasing wealth as one moves away from the city centre \cite{burgess_growth_1925}. People were seen to move up the social hierarchy by moving out away from the city via processes of succession and filtering, while the inner-city was the locus for in-migration and the working class \cite{burgess_growth_1925}.

% transport and development
The private car was a catalyst that took these ongoing trends, and put them into overdrive. With the Model T, Henry Ford and his company initially developed a machine that was only affordable for the wealthy, but throughout the early decades of the 20th century, it become increasingly affordable relative to the yearly incomes for many households \cite{kunstler_geography_1994}. The benefits of having a car were clear, it increased people's mobility and accessibility. They could travel to destinations more quickly, and it increased the number of possible destinations that could be reached in a reasonable travel time. Despite economic downturns, like the great depression or the oil crisis in the 70s, auto production and use continued to grow throughout the 20th century. Practically all of the developed world, and an increasing part of the developing world has become integrated into a system of automobility. By the end of the 20th century, it was estimated that one billion cars were manufactured during the past century \cite{urry_systemautomobility_2004}. It is now only the densest cities and the poorest rural areas in which cars are not a necessity for daily life. Automobility is also inherently linked with many other industries such as parts manufacturers, civil engineering, retail environments, urban planning, oil and gas sectors, and the wealth of nations generally (particularly oil producing nations). If driving declined, it would have a number of effects in the economy and the aforementioned industries would have to close down or pivot. Such economic changes could have various ripple effects in the social geographies of some cities, similar to de-industrialization of what is now called the rust belt of the USA.

% cars and subrubia
Cars provided a form of unfettered mobility that provided residents the ability to live at further destinations from their daily activities. This was a much greater land area, or catchment area, than what could be reached by trains or streetcars, which were confined to very specific rail corridors and by walking distances from stations. Indeed, many scholars have discussed theoretically, as well as shown empirically that cars are a major driver of low-density suburban development \cite{baum-snow_did_2007,glaeser_sprawl_2004}. The regression models by \citeA{baum-snow_did_2007} imply that across a sample of metro areas in the USA, central city populations would have grown by about 8 percent on average had the entire interstate highway system not been built, instead of declining by 17 percent overall. Suburban development increased rampantly during the post-war period, as modern suburban development became affordable for the growing middle-class (particularly the large Baby Boom generation), who had a desire for suburban living, as well as higher incomes allowing them to get such space \cite{anas_urban_1998}. Particularly in the North American context, many local governments were happy to open up such land to private development as well as offered mortgage subsidies (e.g. via tax credits) for home-buyers to spur the process \cite{blais_perverse_2011,ewing_compactness_2015,bruegmann_sprawl:_2008}. Overall, these factors resulted in a more homogeneous sprawl that was often only confined by strict topography such as a water body or mountain range. 

% suburb causes are complex
Causes of suburbanization cannot be summed up by a simple linear equation. There are lots of linked effects (e.g. home owner credits, highway development policy, increasing crime in cities as people leave pushing more to want to move, socio-cultural opinions of what consists a "good life", etc.). Many of these causes are related and have feedback effects. For example, from a transport perspective, we can argue that cars and highways led to urban sprawl by allowing development in formerly inaccessible lands \cite<e.g.>{glaeser_sprawl_2004}, but then this suburban development led to more cars since it made transit less efficient and accessible. This then focused further development patterns on highways as well as low-density housing and retail, further making the car more desirable, \textit{ad infinitum}. Transportation planners, despite being complacent in suburban sprawl, have long understood that urban development is based on feedbacks between transportation and land-use \cite{wegener_land-use_2004}. Furthermore, there is often path dependency and mutual interests \cite{blais_perverse_2011,kunstler_geography_1994} such as the highway industrial complex.

% benefits of suburbs
The massive growth in suburban development over the past 100 years indicates an overwhelming preference and demand for suburban living. These preferences include greater volumes of interior living space, private yards, and access to both the city and the countryside (given that a household has a car). There are also socio-cultural pushes towards suburban living evident in media outlets, and the size of one's home and car are status symbols in many communities \cite{kunstler_geography_1994}. Ideals of suburban living stemmed from utopian visions of cities such as Howard's "garden city", which was a balanced spatial pattern of residences and other human uses (work, retail, services), surrounded by greenbelts and connected by rail lines. Another example is Frank Lloyd Wright's "Broadacre city" which was a grid of single detached homes based on the idea that people were not meant to live as "degraded machines" in high density urban environments \cite{wegener_land-use_2004}. These were based on concepts that  human societies should be more connected to nature and away from the negative effects of cities such as density, noise, and pollution (indeed there is much more recent evidence on how access to green space positively effects well-being). Some suburban developments are able to provide this access and opportunity. These aforementioned benefits of living space and access to nature (and away from hyper-density of human activity) are why many prefer to live in the suburbs today. Certainly, many households make a trade-off between location (i.e. accessibility) and housing quality when deciding where to live \cite{lee_neighborhood_1994,alonso_location_1964}

% Environmental effects
The negative effects of modern suburbaninzation can be grouped into two; environmental and social. The environmental effects are clear. Suburban environments are directly associated with increased driving \cite{ewing_travel_2010,moos_suburban_2015} and with this, comes increased congestion, pollution, and emissions that impact local population health and contribute to climate change. Auto-based transportation is the single most important cause of environmental resource-use \cite{urry_systemautomobility_2004}. This is a combination of the scale of material, space, and power used in the manufacture of cars and car-focused built environments. Transport accounted for 1/3 of C02 emissions during 20th century \cite{urry_systemautomobility_2004}. The life-cycle environmental costs of auto production span the globe across various parts and assembly chains. There are also ample negative environmental effects associated with modern auto-oriented landscapes that cars have enabled, such as larger homes on larger parcels of land relative to the city. The relatively larger housing found in typical suburbs requires more energy use, both in terms of space for heating and cooling, as well as more material goods being produced to fill said space, resulting in larger carbon footprints than smaller urban homes. Moreover, modern suburbanization also takes up more land per person than older urban environments. Much of this was former green space and agricultural land, land uses that are now located further from city centres, meaning food has to be shipped a greater distance. This increased space also requires more money and material devoted to utilities to serve these neighbourhoods (e.g. stormwater sewers and retention ponds, waste water sewers, water transmission lines, roads and highways, etc.). A number of researchers have noted that these environmental costs are not properly incorporated into the costs of driving and land development \cite{blais_perverse_2011,ewing_compactness_2015,kunstler_geography_1994}. In other words, society has been subsidizing environmentally regressive suburban development, instead of designing healthier, more sustainable, and more affordable communities.

% Social Impacts - health and social capital
The negative social impacts of suburbanization are more varied and nuanced. The first of these relate to human health. For example, as more people drive, there has been a substantial increase in traffic fatalities (e.g. more Americans have died in car crashes since 2000 than American soldiers who died during WW2) as well as longer emergency response times \cite{ewing_compactness_2015}. Suburban auto-oriented environments can also quell other, more active, modes of travel such as walking, cycling, and transit (this typically includes walking to and from stops). There is a large body of research that has shown that suburban auto-oriented environments are linked with less physical activity and also with higher rates of obesity, heart disease, and cancer prevalence \cite{ewing_compactness_2015,ewing_relationship_2003}. 
There is also increasing concern that, because of their auto-dependency and low density attributes, suburban environments can cause transport disadvantage, inaccessibility, and barriers to daily travel---particularly for those who do not have regular access to a private vehicle \cite{roorda_trip_2010,lucas_transport_2012,allen_planning_2020}. Similarly, others have argued that reliance on a car and the banality of suburban environments can limit social interactions and creates a barrier to building social capital by creating an overly individualized atmosphere \cite{jacobs_death_1961,kunstler_geography_1994}. My dissertation research focuses on the transport-related social impacts of suburbanization, and how these have developed over time.






\section{The Changing Social Geography of Cities}

Alongside built environment changes, the demographic and socio-economic patterns of regions change over time as well. These regional changes can be dependent on external economic forces. A globalization of the world economy has shaped the economy of cities. This is arguably increasing a wage-gap. In the developed world, particularly in North America, there has been a decline in industrial and manufacturing jobs due to a global re-distribution of manufacturing sites. These jobs were predominantly blue collar and middle class, and during much of the 20th century, most households in North American cities were working, middle-class families. In some cities, such as Toronto, this has been partly offset by an increase in knowledge, office-based, professional work \cite{walks_social_2001}. The result is a polarization of the labour market in terms of occupation and income. Many low-income jobs are in retail or food industries, which have low wages and low employment benefits, particularly for temporary workers. 

In some cases, this polarization of the labour market has resulted in divisions of urban space \cite{walks_income_2013, walks_social_2001}. Higher-income households can always outbid lower-income households for housing quality and preferred locations. If a lower-income neighbourhood has characteristics that a higher-income group finds desirable, gentrification can occur and the original residents may be forced to move due to costs of living increases. The opposite can also occur. Some neighbourhoods, once popular among middle- or higher-income households, can fall out of favour and property values fail to keep up with other neighbourhoods. Over time, lower-income households replace middle- and higher-income households in these neighbourhoods, via what can be broadly be termed as a filtering process \cite{hulchanski_three_2010,ades_are_2012}. Housing is also constrained by zoning. Social housing and aging rentals are clustered in certain parts of cities, which is filtered down to lower SES groups \cite{hulchanski_three_2010}. In Toronto, social housing is relatively dispersed. Many low-income rentals are in the form of Corbusian apartment towers built in the mid-20th century \cite{august_gentrification_2018}. These are mixed throughout the city as well. Those in the inner-suburbs are typically at the bottom of the housing ladder \cite{hulchanski_three_2010,ehrenhalt_great_2012}. Generally, there are very few apartments and lower-income housing in wealthier single-detached neighbourhoods mainly due to NIMBYism and zoning constraining intensification \cite{scally_democracy_2015}. In the case of Toronto, zoning restrictions have focused brownfield development into a small portion of the city's land area, and the most profitable form of development are condominiums or offices. 

Changes in regional socio-economic structure can also be due to demographic transitions, such as aging populations, families having fewer children, and more women in the workforce \cite{bourne_changing_2001,bourne_are_1989}. These can result in changes in income inequalities. For example, increased education rates leave a minority behind with a greater gap in wages. Aging populations can also increase the percent with no yearly income, due to income being the primary measure of inequality \cite{bolton_growing_2012}. Immigration is also a factor \cite{bourne_changing_2001}. Most immigrants are typically of relatively lower SES than current residents and settle in the most affordable parts of cities, but often where there are still some socio-cultural connections, either through social networks or settlement services (this can be thought of as a form of self-selected segregation). 

% urban decline
Suburban housing and lifestyles can provide levels of comfort for many that were not achievable generations prior. This upward mobility to better housing was only achievable for those of certain levels of household income. In many cities during the 20th century, this resulted a hollowing out of the inner-cities, leaving only the poor behind. In this US, this has been termed "white-flight" given the racial correlates with income, or more systematic forms of racism (e.g. exclusionary zoning to limit low-income, often black, households to settle in a community). This concentration of poverty can have further feedback effects, such as increase in crime and urban decay, barriers to social mobility \cite<e.g.>{chetty_effects_2016}, and limited resources due to a declining tax-base \cite{wilson_truly_2012}. Jobs and other resources also became harder to obtain because of greater spatial distances. This has been termed as a "buffer" effect \cite{wilson_truly_2012} or as spatial mismatch \cite{holzer_spatial_1991}, which is when the movement of people and firms from central city areas to the suburbs causes growing employment problems for people who live in inner-cities (especially, low income, and in the United States, black residents).

% urban regeneration
Some of these formally deprived inner-city areas have undergone gentrification. Gentrification can be defined as the replacement of the existing population in an area by one enjoying a higher socioeconomic position \cite{freeman_displacement_2005,bruegmann_sprawl:_2008,lees_gentrification_2013}. It also typically includes the retrofitting of physical urban space (e.g. buildings, the public realm), while maintaining a certain historical character. Generally a neighbourhood is thought of as gentrifying if its SES increases at a faster rate than the rest of the city on average. It usually involves the transition of inner-city neighbourhoods from a status of relative poverty and limited urban investment into an area of greater commodification, reinvestment, and less relative poverty. Gentrification is thus a combination of social change, economic change (e.g. businesses, housing markets), and physical change of the built environment \cite{rose_rethinking_1984,ley_alternative_1986,hammel_model_1996,hamnett_blind_1991}.

The modern, back-to-the-city, form of gentrification began in the 1960s and was soon discussed in depth by researchers in the 1970s \cite{sumka_neighborhood_1979}. Gentrification represents part of a demographic inversion \cite{ehrenhalt_great_2012}. Previously, city structure was adequately explained with urban theories from the Chicago school's concentric circles \cite{burgess_growth_1925}, as well as theories of land rent \cite{alonso_location_1964}. With the onset of gentrification, city centres were becoming wealthy, going against this core-periphery gradient model of the city.

Early explanations of gentrification can generally be divided into production and consumption theories.

The production side argument focuses on supply \cite{hamnett_blind_1991}, and in particular capitalist land markets \cite{smith_gentrification_1987}. The supply side actors include builders, developers, landlords, mortgage lenders, government agencies, and real estate agents - many of whom are trying to maximize their profit.
\citeA{smith_gentrification_1987} frames gentrification using a Marxist perspective of uneven development, which can be defined as how societal and economic development does not take place everywhere at the same speed or in the same direction. \citeA{smith_gentrification_1987} argues that much of this is due to capitalism, that will result in development where costs are lower and the potential for profit is greater. This is termed a \textit{rent gap} in the context of gentrification, which is the difference between the potential profit of re-development and its current use. When the gap is great enough, redevelopment will likely occur given the probability of profit. Throughout the mid-20th century, the movement of capital into the suburbs led to less investment of the inner city, and as the rent gap grew, inner-city investment began to occur. In other words, gentrification can be thought of as a process of creative destruction.

However, several authors have argued that Smith's Marxist explanation is too deterministic and parochial \cite{rose_rethinking_1984,ley_alternative_1986,hamnett_blind_1991}. For example, \citeA{rose_rethinking_1984} argues that Marxist theories are overly "structural" in the role of capitalism in producing gentrification, and does not consider other components. Indeed supply and demand are symbiotic, production of supply (or production of gentrification) won't occur without the demand. In the words of \citeA{hamnett_blind_1991}, "the rent gap may provide the means, but not the motive for gentrification" \cite{hamnett_blind_1991}. This leads to the demand, the consumption side argument of gentrification.

The consumption theories consider a number of factors that are reduced to increased demand for inner city living. The first of these pertain to demographic change. In the 70s, there was an increase in in the population of 20-30 year olds due to baby boom generation, who were entering the housing market. These new families were typically decreasing in size (e.g. number of children per parent), and there were also more women entering the labour market. These changing household structures (more smaller households, more women in the work force), meant there was a greater demand for smaller housing units such as condos, which are generally located in more central areas, areas that are often discussed as gentrifying. As well, there are more gay couples living together, couples without kids, non-related people living with each other (roommates), singles, single parents, elderly, all not desiring traditional suburban homes \cite{ley_alternative_1986,bourne_changing_2001}. Much of the supply for non-traditional homes are in so-called gentrifying neighbourhoods (this is arguably a failure of land use planning and shortsighted development).

The preferences and constraints of these new adults were different than their predecessor cohort. In general, there were increased value in urban amenities among younger generations, compared to previous generations who valued greater living space \cite{ley_alternative_1986,bourne_are_1989}. These urban amenities included access to non-work activities such as cultural, commodities/retail, community. Or in other words, access to diversity in a broad sense, many of the positives of urban living that have been discussed by \citeA{jacobs_death_1961}. Early stages of gentrification were often attributed with counter-cultural lifestyles (e.g. artists, gay communities, activist organizations). There were also preferences for aesthetically pleasing landscape, that could partly be a backlash of the banality of suburbia that many gentrifiers grew up in \cite{hamnett_blind_1991,kunstler_geography_1994}. There was also an important role of individuality and growth of a more sensuous and aesthetic philosophy among the newer middle class (a new leisure class in pursuit of self-actualization, rather than a stable life with a house and a car) \cite{hamnett_blind_1991}. Similarly, some people had preferences for active travel, or car-free lifestyles \cite{schwanen_what_2005}, that have been shown to impact residential location choices in more central areas \cite{cao_how_2016}. 

There are also two other macro-level factors worth noting. One is that the retrofitting of city centres is spurred by governments (e.g. re-designing streets, public parks, plazas, etc.) that increases desirability and demand for these areas \cite{jones_transit-oriented_2016,zuk_gentrification_2018}. For example, improved or new transit infrastructure can increase accessibility and desirability, and thus demand and competition for space \cite{higgins_forty_2016}. In some cases, this has been cited as a catalyst for gentrification \cite{jones_transit-oriented_2016,padeiro_transit-oriented_2019,grube-cavers_urban_2015}. Second, over the past several decades, there have been increases in commuting costs (both in terms of distance because of sprawl, travel times because of congestion, as well as increased fuel costs), that discourage living in distant suburbs.

In sum, gentrification should not be thought of as having a single cause (none of the aforementioned alone), but a combination of the above, that likely interact and feedback to produce the construct that's called gentrification. In general, gentrification is not a simple cause and effect process. Gentrification, and urban change more generally, is a symbiotic process of supply and demand, and of mutual causality. 

The effects of gentrification are both substantial and controversial. First, gentrification makes real changes to the physical environments of cities. Shops and storefronts change to cater to new populations. Streets are upgraded. Buildings are retrofitted. New buildings are built. In some cases, decaying remnants of industry have been converted into office, retail, and residential spaces (e.g. the Distillery District is a prime example of converting former industry land into a bastion of haute commerce in Toronto). As well, in general, living in gentrifying neighbourhoods promotes a more sustainable lifestyle compared to suburbia, it highlights desire for urban living, and pride people have for their non-suburban neighbourhood (before, during, and/or after the gentrification process). Daily behaviour in urban gentrified neighbourhoods have a lower carbon footprint compared to more suburban locales of similar SES \cite{ewing_compactness_2015}. However, gentrifying neighbourhoods may also be increasing commutes for low-income workers who cannot afford to live in these neighbourhoods, and are thus commuting greater distances.

The reciprocal to inner-city gentrification is socio-economic decline in suburbs. For example, in large cities in Canada, the most affordable parts of cities are now found in the inner-suburbs, particularly where has been substantial city-centre gentrification and increased demand for older neighbourhoods near the core. This is evident in studies using census data showing an increase in low-income residents and recent immigrant settlement in the suburbs in Canada \cite{bourne_changing_2001,ades_are_2012,breau_pulling_2018}. Similar trends are occurring in cities in the United States \cite{ehrenhalt_great_2012,delmelle_differentiating_2017}.

Decline in middle-income jobs and increases in wage and income inequality have also resulted in a greater range between the costs of housing, and what lower-income households can afford, particularly within central parts of cities that have greater demand \cite{walks_social_2001}. This has resulted in some lower-income households being priced out of central parts of these cities, and subsequently re-locating to less dense, but more affordable neighbourhoods. Spatial patterns of recent immigrant settlement have also witnessed shifts from central areas towards relatively more affordable suburban environments \cite{ley_relations_2000}. New, recently built, suburban housing stock remain predominately designed for middle to upper income households, typically for traditional (i.e. nuclear) families. But low-income residents are not primarily located in typical suburban housing stock (e.g. single detached homes), but instead, are often clustered in apartment buildings in more suburban locations \cite{cooke_suburbanization_2015,skaburskis_filtering_2014}. In Canadian cities, many of these were designed in an auto-oriented, Corbusian "Tower in the Park" style in the mid- to late-20th century.  There are concerns that these buildings have few on-site services and amenities, lack rent controls, and many consist of decaying building structures; problems that are exasperated by an increased financialization of ownership that focuses on extracting value rather than the well-being of tenants \cite{august_gentrification_2018}. Despite these concerns, they are now the most abundant form of available lower-income housing in many cities due to increased demand, low supply, and high costs of housing in central areas.

From a research perspective, empirical evidence on suburbanization of poverty is usually based on longitudinal analysis of census data or specific panel surveys analyzing changes in the spatial distribution of low-income households or other SES variables over time. Recent research has highlighted trends of suburbanization of poverty in early-industrializing cities in nations such as the United States
\cite{kneebone_suburbanization_2010,howell_racial_2014,cooke_suburbanization_2015}, the Netherlands \cite{hochstenbach_gentrification_2018}, Sweden \shortcite{hedin_neoliberalization_2012}, Scotland \cite{kavanagh_is_2016}, and Australia \cite{randolph_suburbanizing_2014}.

Within the Canadian context, there is evidence of increasing income inequalities both within and between regions \cite{maclachlan_measures_1997,walks_income_2013,bolton_growing_2012,breau_rising_2015,chen_why_2012}, as well as concentrations of low-income households forming in some inner-suburban neighbourhoods \cite{ades_are_2012,pavlic_declining_2014,ades_is_2016,breau_pulling_2018}. For example, \citeA{pavlic_declining_2014} found that there has been decline or stagnation in prosperity in inner-suburban areas relative to central areas and newer outer-ring suburbs. Research by \shortciteA{ross_dimensions_2004} and \shortciteA{ades_are_2012} included spatial segregation indices for low-income households and households in poverty in Canadian cities, finding that economic segregation is increasing at a neighbourhood level and that low-income households are becoming increasingly concentrated and isolated within their spatial units. Statistical mapping exercises have also highlighted that neighbourhoods with higher concentrations of low-income households have been drifting further from the downtown core over the past several decades \cite{ades_are_2012,breau_pulling_2018}.

Overall, these trends of suburbanization of poverty have raised concerns regarding the increasing polarization and segregation of urban space by class and income \cite{hulchanski_three_2010,walks_income_2013,ades_are_2012}, as well as negative impacts caused by eviction and displacement on individual well-being and disrupting community cohesion \cite{august_challenging_2014,august_its_2016}. As such, it is important, yet understudied, how households end up to be poor and live in the suburbs. Chapter \ref{ch:pathsubpov} of this dissertation directly examines the propensities of different pathways to suburban poverty, specifically to what extent is suburban poverty a product of moving from city centres, immigration, or becoming and remaining poor in-place.


\section{Neighbourhood Effects \& Daily Activity Behaviour}

Studying neighbourhoods and how they change over time is important because of the beneficial or adverse effects living in certain neighbourhoods (i.e. neighbourhood characteristics) can have on individual outcomes. This is often termed as "neighbourhood effects" \cite{sampson_assessing_2002}. In a literature review, \citeA{sampson_assessing_2002} argues that there are four classes of neighbourhood mechanisms that can impact social outcomes:
1) Social ties / interactions. i.e. the level of social capital, which is generally conceptualized as a resource that is realized through social relationships.
2) Norms and collective efficacy, which is that the willingness of residents to intervene on behalf of children may depend on conditions of mutual trust and shared expectations among residents. i.e. can be considered as mutual trust and social cohesion.
3) Institutional resources, particularly in terms of quality, quantity, and diversity of such that meet the needs of the community (e.g. libraries, day care, schools, health, etc.). This can be measured by accessibility (opportunity) and by participation (outcomes).
4) Daily activity patterns and exposure to different components of society and the built environment. To this final point, I would also add accessibility and relation to the city as a whole (which is not mentioned by Sampson). My research is focused on the third and fourth points highlighted by \citeA{sampson_assessing_2002}.

% From a research perspective, studying neighbourhood effects can involve asking whether a particular type of community context relevant to a behaviour of interest, what is the geographic scale of that context, which dimensions of that context are likely to be influential, and how specific contextual dimensions fit into a larger causal understanding of a behaviour \cite{lee_neighborhood_1994}.

% concentrated poverty
Importantly, it is well observed that decreases in neighbourhood social indicators (e.g. increased crime, poverty, etc.) result in negative impacts for  residents  \cite{sampson_assessing_2002}. Poverty concentration can also have feedback effects \cite{wilson_truly_2012}, that can be termed concentration effects and buffer effects. Concentration effects refer to the constraints and impacts associated with living in a neighbourhood in which the population is overwhelmingly socially disadvantaged. Buffer effects, or lack thereof, refers to whether there are a sufficient number of working- and middle-class professional families in a neighbourhood to absorb the shock or cushion the effect of uneven economic growth and periodic recessions on poor neighbourhoods. For example, if the poor become increasingly segregated, it often becomes more difficult to plan and provide needed services due to declining tax base, meaning less services to alleviate poverty. As well, poverty concentration can lead to limited economic development in poor areas - e.g. grocery stores or other services may not want to locate in a poor area as it might be less profitable \cite{ades_are_2012}. The aforementioned has partly motivated research on spatial segregation \cite{massey_dimensions_1988}, in order to highlight areas that are more likely to be at risk.

% social mobility
Where you live can also impact long-term social mobility. One example is the Moving to Opportunities study, that provides excellent evidence that a child's exposure to better environments impacts their long term social mobility. In this study, 4,604 families in low-income neighbourhoods were randomly assigned into 3 groups: 1) Received subsidized housing voucher if they moved to a census tract with a poverty rate below 10 percent. 2) Received subsidized housing voucher without any additional contingencies. 3) A control group without a voucher. Then the authors estimated "intent-to-treat" effects based on long-run tax records. These effects are essentially differences between treatment and control means (i.e. a difference-in-difference model). These are estimated as dummy variables in an OLS modelling effects on income and educational attainment, and the models include both individual and neighbourhood controls, and clusters errors at the household level \cite{chetty_effects_2016}. Their main finding is that children (under 13) who moved to higher SES neighbourhoods have better long-term outcomes in terms of education and income.

% could add more detail here specifically about the effects of gentrification, one of my comps answers has some stuff already written

Another important neighbourhood effect, only touched upon above, is how one's built environment can facilitate or act as a barrier to daily travel and activity participation. Indeed, travel is a derived demand, derived from wanting to participate in daily activities \cite{jones_behavioural_1979,buliung_activitytravel_2007}. From a rational point of view, someone takes a trip when the utility (or value) of an activity outweighs the costs of travel (where costs can be considered as a generalized combination of monetary, temporal, and other factors). The characteristics of daily travel can be described in a number of dimensions: These include the number of trips per day, the activities participated at the destinations of these trips, the time and distance travelled on each of the trip legs as well as in aggregate, the mode of travel utilized, and which route people take on their journey. Daily travel has two primary predictors, individual factors and environmental factors \cite{hanson_determinants_1982}. 

% individual
An individual's travel pattern is based on how they chose to participate in certain activities at certain places at certain times of day. Travel can thus be understood as the movement required to participate in daily activities \cite{hanson_determinants_1982}. Different people have different preferences for or constraints on the activities in which they participate (e.g. school, work, shopping, health care, etc.) that are related to their demographics and socio-economic status. For example, the elderly are less likely to travel during morning commuting periods than the middle aged. Indeed, the majority of quantitative travel behaviour studies include variables on age, employment status, and gender \cite{hanson_determinants_1982}. They also include variables pertaining to mobility tools and ownership, that vary by person and household. Examples include car ownership or whether or not someone has at transit pass. These tools if available can facilitate travel. Income is thus a factor as well. For example, limited income can then create barriers to travel due to unwillingness or inability to afford a car, bicycle, or pay for regular transit trips.

% preferences
Travel behaviour is also based on personal preferences. For instance, research by \cite{schwanen_what_2005} shows that people who have a preference for living in walkable environments and who also live in walkable environments are more likely to walk than people who have a preference for suburban environments but live in walkable environments. Preferences for other life choices can also have an indirect effect on travel. For example, residential self-selection can be a confounding factor in travel behaviour studies - i.e. endogenous to the behaviour \cite{cao_how_2016}.

% built env
The built environment also plays an important role in facilitating travel and activity behaviour. The built environment can be conceptualized into two scales; urban form and urban structure. Urban form is usually discussed in terms of local contexts---what is in the nearby (walkable) vicinity of a household. Components of urban form include streets, buildings, parks; architecture and urban design at a more general level. Indeed, there is ample research showing how "good" built environments can result in more sustainable outcomes such as increased transit use and lower vehicle kilometres travelled \cite{ewing_travel_2010,ewing_compactness_2015}, where "good" in this case often refers to the 3Ds; \textit{density} of activity, population, employment, etc.\textit{diversity} of land uses, often measured in terms of entropy, and the \textit{design} of street networks, size of blocks, and available sidewalks \cite{cervero_travel_1997}.

Urban structure pertains to how a location is situated within the context of an urban region, and how characteristics of locations are related to each other through networks \cite{anas_urban_1998}. Urban structure is often framed in terms of accessibility, which can be summarized as the ease of reaching activity destinations \cite{hansen_how_1959,geurs_accessibility_2004}. There are a number of different flavours of accessibility pertaining to different destination activity types, times of day, modes of travel, or spatiotemporal constraints \cite{geurs_accessibility_2004,levinson_towards_2020}. A meta-review by \citeA{ewing_travel_2010} found that destination accessibility is strongly related to VKT (Vehicle Kilometres Travelled), roughly the same as first three Ds combined. 

Accessibility measures can be used to predict a number of forms of travel behaviour. Indeed it has long been observed that "the frequency of human interactions such as messages, trips or migrations between two locations (cities or regions) is proportional to their size, but inversely proportional to their distance" \cite{wegener_land-use_2004}. This is analogous to the gravitational understanding of accessibility, and more practically, as a basis for spatial interaction modelling, which can be used to predict travel flows within a region. As well, accessibility can be used to predict a variety of travel behaviour outcomes such as mode choice, activity rates, travel times, and travel distances \cite{ewing_travel_2010,koenig_indicators_1980,badoe_transportationland-use_2000}. 

Notable as well is that environmental and socioeconomic factors can have varying effects depending on the travel behaviour outcome being studied. Research by \citeA{hanson_determinants_1982} finds that SES and land use impact all forms of travel behaviour measured. But, SES is main driver of trip rates and participation in certain types of activities, while the built environment is found to be the main driver of travel distances. There can be varying and interactive effects. For example, lower income and zero-car households were much more sensitive to having transit accessibility increasing trip rates than not being part of this overlapping set \cite{allen_planning_2020}.

These matters of travel behaviour are quite important for transport planning as they allow planners to examine and predict how different interventions would fare when it comes to improving these outcomes. These are often framed in terms of economic or environmental benefits. For example, greater levels transit accessibility are correlated with higher levels or transit ridership, lower VKT, and reduced travel times \cite{badoe_transportationland-use_2000,ewing_travel_2010}. As such, many transport planners have called for improving levels of accessibility, and further including accessibility measurement into transport planning, in order to achieve outcomes across these dimensions \cite{handy_measuring_1997}.

% neighbourhood inaccessiblity and transport releated social exclusion
There is also an increasing concern that neighbourhoods with limited accessibility to daily activities (e.g. grocery stores, health services, employment, etc.) can have negative effects for residents. In the worst cases, this can result in social isolation and social exclusion \cite{ades_are_2012,lucas_transport_2012,lucas_is_2018}. Travel requires costs (e.g. money to buy a car, time spent travelling). Travel costs are generally felt to a greater degree by lower-income households (e.g. owning or leasing a private car is a luxury that many lower-income households cannot afford). Transport disadvantage (e.g. such as not having access to a car, low levels of public transit service, etc.) can admix with other forms of social disadvantage (e.g. unemployment, low income, etc.), an effect termed as transport poverty \cite{lucas_transport_2012, allen_sizing_2019}. This can result in adverse outcomes such as reduced accessibility, lengthy commutes, and lower levels of daily activity participation. This can then result in social exclusion, and feedback to abet transport and social disadvantage. For example, being unemployed and not having a car while also living in an area poorly served by public transit can limit one's ability to travel to employment opportunities, prolonging unemployment. This prolonged unemployment can result in further levels of social and/or transport disadvantage. 

\begin{figure}[h]
	\centering
	\caption{Conceptualization of transport poverty adapted from Lucas (2012)}
	\includegraphics[width=3in]{figures/tpov.png}
	\label{tpov}
\end{figure}

Transport poverty can be heightened in poor neighbourhoods with low levels of accessibility \cite{lucas_is_2018, allen_sizing_2019}. For example, this can occur if jobs in city decline, but poor population does not, many will have to try to find work in the suburbs, increasing commuting costs. This has been termed spatial mismatch \cite{holzer_spatial_1991}. As well, finding employment in suburban areas can be more difficult since search costs are greater, particularly if using more informal search networks (e.g. word of mouth, etc.). Moreover, if suburban populations have increased, then wages may decline at suburban job locations since there are more people competing for nearby jobs. If wages are not able to supersede commuting costs, then people will not take these jobs. This can also be compounded with other components of social disadvantage experienced by poor inner-city residents (lack of education, lack of social networks, single parenting, etc.) \cite{holzer_spatial_1991}. It is arguable that this spatial mismatch is becoming inverted, with poverty concentrated in the inner-suburbs but jobs being concentrated in the city centre or dispersed in auto-oriented business parks. Chapter \ref{ch:subtrapov} of this dissertation examines whether, and if so how, both social and transport disadvantage are increasing in more suburban neighbourhoods. i.e. are there trends of a suburbanization of transport poverty







\section{Transportation Planning \& Demographic Change}

% tranasport planning and equity

A primary goal of urban transport is to provide people the ability to travel to daily activities in a reasonable amount of time \cite{martens_transport_2016}, and especially to try to reduce the prevalence of transport poverty. If people are not able to travel to activity destinations necessary for their well being it indicates that the transportation network is not fulfilling one of its intended purposes. 

Despite these basic objectives, it has been argued that the traditional paradigm of transport planning is socially regressive \cite{martens_transport_2016}. Transport planning practice was historically based on the observation that aggregate patterns of travel behaviour are the product of supply (of activities) and demand (population) who are limited by constraints (money, mobility, nearby available opportunities) \cite{hanson_determinants_1982,wegener_land-use_2004}.
This historically took the form of four-step travel demand modelling, and in some regions, has evolved into more detailed activity-based micro-simulations \cite{wegener_land-use_2004}. A result of traditional transport planning models is an egalitarian form focused on the network (and usually road and highway networks) rather than individuals. Travel demand models are most often built on revealed preference survey data, and are thus biased towards those groups and individuals who travel more often and further distances. Those who travel less are not considered as much due to the summation on links. Wealthier people tend to drive further distances by car than the poor, and are part of more links on the network model. Therefore, this type of transport planning benefits the rich more than the poor, and thus prioritizing investments in such a way creates further inequalities \cite{martens_transport_2016}.

% 	Planning for equality in mobility by car can be socially regressive. For one, such planning rarely considers people without cars (e.g. because they cannot afford a car, are not physically able to drive, or of age to drive). The result of prioritizing car-based travel can then feedback into encouraging people travelling more by car, if they can, but then further exasperates inequalities between those who can and cannot drive \cite{martens_transport_2016}. It also further abets modal inequalities, prioritizes streets to be focused on the car, and erodes the urban fabric, in which many lower-income people depend. Many urban neighbourhoods were cleaved with highways throughout the 20th century, destroying the communities through which they passed, either directly by evicting people and tearing down housing to make way for the highway, or by fracturing the neighbourhoods adjacent to them \cite{jacobs_death_1961}. 

% planning for accessibility philosophy
As such, several researchers have argued for transport planning to be based on philosophical arguments and theories of social justice \cite{banister_inequality_2018,pereira_distributive_2017,martens_transport_2016}. Development of a theory of justice for transportation has been thought of as a two-step process. First this has involved setting the good of transportation as separate from all other goods (\citeA{martens_transport_2016} argues this via Walzer's Spheres of Justice). Second, this has involved outlining principles of justice that can define and distribute this good. Recent discussion on distributive justice of transport has drawn upon primarily from two philosophical ideas. The first is \textit{Rawls' egalitarianism} \cite{rawls_theory_1971}, that argues that people should have as much freedom as possible, as long as it does not impact other people's freedom (e.g. cars should not limit people's ability to walk in a city). In other words, inequality is only fair if people have equality of opportunity and if policy benefits the least advantaged members of society. i.e. inequality is okay if they stem from preferences, not arbitrary circumstance. Since inequality is inevitable, policy should be based on a \textit{maximin} criterion, that policy should maximize the minimum level of primary goods of the people in the worst-off position \cite{martens_transport_2016}. The Rawlsian approach focuses on goods, such as income. The second, and expanding on the Rawlsian approach, are \textit{Capability Approaches} where the focus should not be on goods, as Rawls' suggests, but on human capabilities. i.e. a focus on the ends (e.g. activity participation), not the means (e.g. income, owning a car) \cite{sen_human_2005}. Reducing capability inequalities will thus increase equality of opportunities. \citeA{martens_transport_2016} and \citeA{pereira_distributive_2017} use this for setting an agenda for transport justice, that we should not be trying to fairly distribute a good (such as income or cars) but instead fairly distribute a capability, and in the context of transport, this capability is accessibility, the ease of reaching activity destinations. They argue that this should be based on setting minimum standards of accessibility to destinations, minimizing inequalities in accessibility, and mitigating transport externalities. 

% what should transport system be?
Many researchers have argued that accessibility benchmarks should be specifically defined in transport planning documents \cite{manaugh_integrating_2015, allen_benchmarking_2019}. In general, accessibility should not be so low that it infringes upon peoples rights, nor should accessibility levels be improved for some if they worsen it for others who already have less. However, it is still an open question of what is an acceptable minimum level of accessibility  (or even how to measure it), and how this varies for different people with varying constraints and capabilities. Accessibility is indeed a combined capability, based on these individual components (e.g. physical ability to ride a bike, income to afford a car, etc.) and the transport-land-use system. Defining sufficient accessibility is difficult, as it requires knowledge of the distribution of existing levels of accessibility and consequences for real people. \citeA{martens_transport_2016} argues this should be based on the relationship between accessibility and activity participation, that likely has a non-linear shape \cite{allen_planning_2020}.  Then the question becomes, what is a fair level of activity participation? Activity participation captures outcomes (what people do), while accessibility captures capability (what people can do). This should be based on the benefits, not just the quantity of activity participation (which is more of a proxy for the quality, but can indicate possible social exclusion).  However, current methods of such are limited to econometric models and travel survey data, and simply counting activity participation by types of activities \cite{fransen_spatio-temporal_2018,allen_planning_2020}. 

%Another uncertainty in defining accessibility thresholds is that even though everyone wants accessibility, they want it to varying extents, and value different flavours of accessibility more or less than others (e.g. some have greater preference for access to jobs, others access to green space, etc.), making sufficient accessibility difficult to define.



\section{Research Gaps \& Motivation}

% relate to urban dynamics
One area to be particular concerned about is how the social geographies in cities are changing over time. As I noted in a previous section, the distribution of the low-SES households has shifted from city centres to more suburban locales \cite{ehrenhalt_great_2012,ades_are_2012}. Given these trends of suburbanization of poverty, there are now more low-income households in more suburban areas than in previous decades \cite{ades_are_2012,ades_is_2016,breau_pulling_2018}. This is potentially increasing the risks of transport poverty, and further exasperating urban inequalities more generally. However, it remains unknown how the risks of transport poverty are increasing in cities alongside trends of suburbanization of poverty. Research on neighbourhood socio-economic change has typically not included transport variables like transit accessibility, car ownership, commute times, and activity participation rates - key indicators of transport poverty and its outcomes. There have been a few studies that descriptively examine changes in public transit accessibility over time alongside changes in neighbourhood level socio-economic status \shortcite{foth_towards_2013,farber_transit_2017,deboosere_understanding_2019}. However, these have been predominantly focused on public transit infrastructure, and did not consider car-ownership, an important component of transport (dis)advantage, nor have they considered outcomes such as activity participation. They are also only focused on a short time period or only consider the impacts of specific transit lines, rather than examining region wide changes over periods longer than a decade. There is some research by transportation engineers on analyzing changes in trip and activity rates over longer periods time \shortcite<e.g.>{roorda_two_2008,kasraian_multi-decade_2020,ozonder_longitudinal_2020}. These studies tend to focus on aggregate trends at a regional level or improving the predictive capability of travel demand models, but they provide little insight on the links between travel behaviour with neighbourhood-level demographic changes, increasing income inequalities, and geographies of transport disadvantage. These gaps the literature will motivate Chapter \ref{ch:subtrapov} of this dissertation, which focuses on analyzing and modelling how the geography of activity participation and commute times changes over time, relative to changes in accessibility and urban social geographies.

Chapter \ref{ch:pathsubpov} is particularly motivated by how ample research \cite<e.g.>{ades_is_2016,breau_pulling_2018,grant_changing_2020} has shown trends suburbanization of poverty at regional and neighbourhood levels in Canada, but has been unable to uncover how residents end up to become poor and living in the suburbs. This is because this existing research, as well as the the analysis in Chapter \ref{ch:subtrapov}, focuses on changing neighbourhoods rather than the individuals and households within these neighbourhoods. While there have been studies looking at residential mobility relating to gentrification \cite{freeman_displacement_2005, ding_gentrification_2016, dragan_does_2019, delmelle_new_2020}, little has examined residential mobility pathways to suburban poverty. The objective of Chapter \ref{ch:pathsubpov} is to thus first develop a conceptual framework describing different potential pathways to suburban poverty, and then uses individual-level data to quantify these pathways across urban regions in Canada. This analysis specifically estimates for the first time in Canada to what extent suburban poverty stems from intra-urban residential mobility, external immigration, and remaining poor in-place---thus providing an important link between theory and empirical evidence on the urban dynamics that lead to suburbanization of poverty. Understanding and quantifying the propensities of these pathways will provide important knowledge about how the social demographics of cities are changing, which leading to contemporary suburban geographies of poverty concentration.

Chapter \ref{ch:lowinctra} is motivated around how some transit rich neighbourhoods have become more expensive (e.g. gentrifying) and are potentially resulting in the residential displacement and exclusion of low-income residents. This has important social inclusion implications since many low-income residents are reliant on public transit for travelling to daily activities \cite{allen_planning_2020,barri_can_2021}, including finding employment \cite{fransen_relationship_2019,bastiaanssen_does_2021}. Moving away from relied upon public transit could thus decrease capabilities for activity participation, and in some cases, increase risks of transport-related social exclusion \cite{lucas_transport_2012,allen_planning_2020}. However, previous quantitative studies have mixed or inconclusive results about whether low-income residents move disproportionately out of gentrifying neighbourhoods or whether newly built transit lines directly leads to the displacement of low-income residents \cite{rayle_investigating_2015, zuk_gentrification_2018,padeiro_transit-oriented_2019,delmelle_transit-induced_2021}. This existing research that studies income inequalities of residential mobility in relation to public transit has focused only on the outcomes of specific transit routes, which serve only a fraction of the population in a region, rather than comprehensively examining changes across an entire region. As such, Chapter \ref{ch:lowinctra} researches whether there are social inequalities at a regional level pertaining to whether lower-income residents are unequally reducing their levels of transit accessibility when they move. Specifically, Chapter \ref{ch:lowinctra} measures changes in transit accessibility experienced by intra-urban movers by income level. 

