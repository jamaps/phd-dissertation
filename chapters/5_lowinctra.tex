\section*{{Abstract:}}


	Lower residents are often reliant on public transit for daily travel. They have also historically concentrated in centrally located neighbourhoods with relatively higher levels of transit accessibility. However, during the late 20th and early 21st century, many urban regions have undergone trends of inner-city gentrification and suburbanization of poverty, raising concerns that low-income residents are disproportionately moving away from relied upon transit service. In this paper, we investigate what occurs when low-income residents change dwellings within a region: do they experience a reduction in their levels of transit accessibility, how does this compare to higher-income movers, and how has this changed over time? We examine this by linking historical transit accessibility measures to annual individual-level panel data representing 20\% of tax filers in Toronto, Canada from 1988 to 2018. We then analyze changes in transit accessibility for intra-urban movers via descriptive statistics and regression models to answer whether there are significant differences in individual changes in transit accessibility by low-income status. We find that low-income residents do, on average, experience reductions in their levels of transit accessibility when moving within the region, but they do not undergo as great of a reduction in transit accessibility as higher-income movers. However, the gap in experienced changes in transit accessibility between high- and low-income movers is converging over time.


\section*{{Keywords:}}

transit accessibility, residential mobility, income, displacement, exclusion



\section{Introduction}

meow